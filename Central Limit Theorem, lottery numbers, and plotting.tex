% This LaTeX was auto-generated from MATLAB code.
% To make changes, update the MATLAB code and export to LaTeX again.

\documentclass{article}

\usepackage[utf8]{inputenc}
\usepackage[T1]{fontenc}
\usepackage{lmodern}
\usepackage{graphicx}
\usepackage{color}
\usepackage{listings}
\usepackage{hyperref}
\usepackage{amsmath}
\usepackage{amsfonts}
\usepackage{epstopdf}
\usepackage{matlab}

\sloppy
\epstopdfsetup{outdir=./}
\graphicspath{ {./Central Limit Theorem, lottery numbers, and plotting_images/} }

\begin{document}

\begin{par}
\begin{flushright}
Scarlett Royle, 201245875
\end{flushright}
\end{par}

\begin{par}
\begin{flushright}
26/10/2018
\end{flushright}
\end{par}

\matlabheading{\underline{PHYS205: Problem set 3}}


\matlabheading{\underline{Question 1a}}

\begin{par}
\begin{flushleft}
For this question, I generated sets of random numbers, plotted their sums on histograms and analysed the distributions in order to demonstrate Central Limit Theorem.
\end{flushleft}
\end{par}

\begin{matlabcode}
x = rand(1000, 1); %1000 random numbers, 0-1

figure
hist(x, 20); %20 bins
hold on
[countsx, binsx] = hist(x, 20);
abs_unc = sqrt(countsx); %absolute uncetainties array
errorbar(binsx, countsx, abs_unc, abs_unc,'.'); %plotting errorbars: x, y, +ve, -ve unc.
xlabel('Random x');
ylabel('Frequency density');
title('Histogram of 1,000 random numbers');
hold off
\end{matlabcode}
\begin{center}
\includegraphics[width=\maxwidth{56.196688409433015em}]{figure_0}
\end{center}

\begin{par}
\begin{flushleft}
\textit{Histogram 1: Histogram of 1,000 random x values between 0 and 1 and their frequency}
\end{flushleft}
\end{par}

\begin{par}
\begin{flushleft}
By increasing the number of random numbers, N, it is expected that the absolute uncertainty ($=\sqrt{N}$) will increase and the relative uncertainty ($=\frac{1}{\sqrt{N}}$) will decrease. This can be demonstrated by comparing the relative sizees of the errors on Histogram 1, where N = 1,000, with those on a histogram where N = 10,000 (\textit{Histogram 2):}
\end{flushleft}
\end{par}

\begin{matlabcode}
x2 = rand(10000, 1); %10,000 random numbers, 0-1

figure
hist(x2, 20); %20 bins
hold on
[countsx2, binsx2] = hist(x2, 20); %assigning the counts and the bin numbers to variables
abs_uncx2 = sqrt(countsx2); %absolute uncetainties array
errorbar(binsx2, countsx2, abs_uncx2, abs_uncx2,'.'); %plotting errorbars: x, y, +ve, -ve
% unc.
xlabel('Random x2');
ylabel('Frequency density');
title('Histogram of 10,000 random numbers');
hold off
\end{matlabcode}
\begin{center}
\includegraphics[width=\maxwidth{56.196688409433015em}]{figure_1}
\end{center}

\begin{par}
\begin{flushleft}
\textit{Histogram 2: Histogram of 10,000 random x values between 0 and 1 and their frequency}
\end{flushleft}
\end{par}

\begin{par}
\begin{flushleft}
It can be seen that, relative to the bin height, the errors on Histogram 2 are smaller than on Histogram 1, showing that for increased N, the relative uncertainty is smaller. We can also demonstrate this by calculating the mean relative uncertainty of each dataset:
\end{flushleft}
\end{par}

\begin{matlabcode}
rel_uncx = abs_unc ./ countsx; %relative uncertainties array for x
rel_uncx2 = abs_uncx2 ./ countsx2; %for x2

mean_rel_uncs = [mean(rel_uncx); mean(rel_uncx2)]; %average relative uncertainties for 
% x and x2
DataSet = {'x'; 'x2'};
N = {'1,000'; '10,000'};

unc_table = table(DataSet, N, mean_rel_uncs);
disp(unc_table)
\end{matlabcode}
\begin{matlaboutput}
    DataSet       N        mean_rel_uncs
    _______    ________    _____________

     'x'       '1,000'        0.14243   
     'x2'      '10,000'      0.044754   
\end{matlaboutput}

\begin{par}
\begin{flushleft}
\textit{Table 1: Table of mean relative uncertainties for x and x2 where N = 1,000 and N = 10,000 respectively}
\end{flushleft}
\end{par}

\begin{par}
\begin{flushleft}
The average relative uncertainty for x2 (N = 10,000 numbers) is smaller than that for x (N = 1,000 numbers), showing that as N increases, the relative uncertainty decreases.
\end{flushleft}
\end{par}


\matlabheading{\underline{Question 1b}}

\begin{par}
\begin{flushleft}
For the following two sections, a function was created to generate sets (of a given size, of a given number) of random numbers and sum them. This was called 'randomSetSum().' 
\end{flushleft}
\end{par}

\begin{par}
\begin{flushleft}
The function can be seen in \textit{Appendix 1} at the end of this report (MATLAB Live editor requires that functions come at the end of the document).
\end{flushleft}
\end{par}

\begin{matlabcode}
[~, sum1] = randomSetSum(1000, 2); %1000 pairs of random numbers (10 sets of 2 numbers)
figure
hist(sum1, 20);
hold on
[counts1, bins1] = hist(sum1, 20);
abs_unc1 = sqrt(counts1); %absolute uncertainties
errorbar(bins1, counts1, abs_unc1, abs_unc1,'.'); %plotting errorbars
xlabel('Sums of two random numbers');
ylabel('Frequency density');
title('Histogram of the sums of two random numbers');
hold off
\end{matlabcode}
\begin{center}
\includegraphics[width=\maxwidth{56.196688409433015em}]{figure_2}
\end{center}

\begin{par}
\begin{flushleft}
\textit{Histogram 3: Histogram of 1,000 sums of pairs of random numbers and their frequency}
\end{flushleft}
\end{par}


\matlabheading{\underline{Question 1c}}

\begin{matlabcode}
[~, sum2] = randomSetSum(1000, 12); %1000 sets of 12 random numbers. The sum of each set 
% stored as 'sum2'
figure
hist(sum2, 20);
hold on
[counts2, bins2] = hist(sum2, 20);
abs_unc2 = sqrt(counts2); %absolute uncertainties
errorbar(bins2, counts2, abs_unc2, abs_unc2, '.'); %plotting errorbars 
xlabel('Sums of 12 random numbers');
ylabel('Frequency density');
title('Histogram of the sums of 12 random numbers');
hold off
\end{matlabcode}
\begin{center}
\includegraphics[width=\maxwidth{56.196688409433015em}]{figure_3}
\end{center}

\begin{par}
\begin{flushleft}
\textit{Histogram 4: Histogram of the sums of 1,000 sets of 12 random numbers and their frequency}
\end{flushleft}
\end{par}


\matlabheading{\underline{Question 1d}}

\begin{matlabcode}
Mean = [mean(x); mean(sum1); mean(sum2)];
Variance = [var(x); var(sum1); var(sum2)];
Standard_deviation = [std(x); std(sum1); std(sum2)];
Data_set = {'Single (x)'; 'Pairs'; 'Sets of 12'};

tableResults = table(Data_set, Mean, Variance, Standard_deviation);
disp(tableResults)
\end{matlabcode}
\begin{matlaboutput}
      Data_set       Mean      Variance    Standard_deviation
    ____________    _______    ________    __________________

    'Single (x)'    0.50667    0.08122          0.28499      
    'Pairs'          1.0036    0.15797          0.39745      
    'Sets of 12'     5.9796       1.04           1.0198      
\end{matlaboutput}

\begin{par}
\begin{flushleft}
\textit{Table 2: Table of results for each data set showing the mean, variance and standard deviation of each}
\end{flushleft}
\end{par}


\matlabheading{\underline{Question 1e}}

\begin{par}
\begin{flushleft}
The shapes of the distribution become more peaked as the number of numbers being summed, N, increases; they appear more Gaussian. As can be seen from Table 2, the standard deviation approaches 1 as N increases.
\end{flushleft}
\end{par}

\begin{par}
\begin{flushleft}
Central limit theorem establishes that, when independent random variables are added, their sum tends towards a normal distribution. As the number being added increases, the standard deviation tends towards 1 and the spread of the curve decreases. This agrees with the observations above.
\end{flushleft}
\end{par}

\begin{par}
\begin{flushleft}
In the physical sciences, this means that the larger the data set collected, the smaller the uncertainty.
\end{flushleft}
\end{par}


\matlabheading{\underline{Question 2:}}

\begin{par}
\begin{flushleft}
'Generate 6 random numbers from 1-49 inclusive for your lottery ticket. Rember that you can't use the same lottery number twice in one draw. Order these smallest to largest.'
\end{flushleft}
\end{par}

\begin{matlabcode}
q2 = randperm(49, 6); %a row vector of 6 random numbers, 1-49 inclusive, of unique 
% values (no repetitions)
q2 = sort(q2); %'sort' function sorts the numbers, smallest to largest
disp('Your lottery numbers are: ')
\end{matlabcode}
\begin{matlaboutput}
Your lottery numbers are: 
\end{matlaboutput}
\begin{matlabcode}
disp(q2)
\end{matlabcode}
\begin{matlaboutput}
     6    13    14    16    22    28
\end{matlaboutput}


\matlabheading{\underline{Question 3}}

\begin{par}
\begin{flushleft}
'Plot the data for photocurrent vs voltage and the uncertainty in voltage given in the Plank\_raw.csv file.'
\end{flushleft}
\end{par}

\begin{matlabcode}
plank_data = csvread('Plank_raw.csv');
V = plank_data(:,1); %voltage
Ip = plank_data(:,2); %photocurrent
dIp = plank_data(:,3); %unc. in photocurrent
n = length(V); %number of entries

figure
plot(V, Ip, '.r'); %plotting the plank data
hold on
ylim([0, 1200]) %setting a limit on the y axis
errorbar(V, Ip, dIp, '.r') %plotting the errorbars. Very small error in y (0.5) not 
% visible
xlabel('Voltage [V]');
ylabel('Photocurrent [nA]');
title('Photocurrent as a Function of Voltage');
\end{matlabcode}
\begin{center}
\includegraphics[width=\maxwidth{56.196688409433015em}]{figure_4}
\end{center}

\begin{par}
\begin{flushleft}
\textit{Plot 1: Plot of photocurrent vs voltage for the Plank data}
\end{flushleft}
\end{par}


\matlabheading{\underline{Appendix 1}}

\begin{par}
\begin{flushleft}
This function was used to generate sets of random numbers of a given size in Question 1
\end{flushleft}
\end{par}

\begin{matlabcode}
function [random, column_sum] = randomSetSum(sets, numbers)
%gives a random number matrix of 'sets' sets of 'numbers' numbers ('random') and sums
%each set ('column_sum')
random = rand(numbers, sets); %rows, columns. Numbers between 0-1 by default
column_sum = sum(random); %the sums of each set
end
\end{matlabcode}

\end{document}
